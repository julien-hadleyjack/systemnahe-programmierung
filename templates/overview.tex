% Warns the user when LateX is wrongly used
% http://www.ctan.org/tex-archive/macros/latex/contrib/nag
\RequirePackage[l2tabu, orthodox]{nag}

% http://www.ctan.org/pkg/scrreprt
\documentclass[
    pagesize=pdftex,    % Page size is set with \pdfpagewidth and \pdfpageheight
    twoside=false,      % Einseitiger Druck.
    fontsize=12pt,      % Schriftgroesse
    parskip=half,       % Halbe Zeile Abstand zwischen Absätzen.
    headsepline,        % Linie nach Kopfzeile.
    footsepline,        % Linie vor Fusszeile.
    abstract=false,     % Abstract Überschriften
    listof=totoc,       % show table of figures in toc but unnumbered
    toc=bibliography,   % show bibliography in toc but unnumbered
]{scrreprt}

\usepackage{special-generated/customsettings}

\title{ {{ title }} }
\author{ {{ author }} }
\date{ {{ date }} }

% Support for hypertext links
% http://www.ctan.org/pkg/hyperref
\usepackage[%
    pdftitle={ {{ title }} },                % text for PDF Title field 
    pdfauthor={ {{ author }} },              % text for PDF Author field 
    pdfsubject={ {{ title }} },              % text for PDF Subject field 
    pdfcreator={pdflatex, LaTeX with KOMA-Script},   % text for PDF Creator field 
    pdfpagemode=UseOutlines,        % show bookmarks
    
    pdflang={de},                   % language identifier (RFC 3066)
    
    pdflang={en},                   % language identifier (RFC 3066)
    
    hidelinks,                      % hide links (no color, no border)
    bookmarksnumbered=true,         % put section numbers in bookmarks
    pdfdisplaydoctitle=true,        % display document title instead of file name in title bar
]{hyperref}

% A new bookmark (outline) organization for package hyperref
% bookmarks are generated on first compile
% http://www.ctan.org/pkg/bookmark
% \usepackage{bookmark}

%%% New commands.

% Don't output references in case they're empty
% http://tex.stackexchange.com/questions/74476/how-to-avoid-empty-thebibliography-environment-bibtex-if-there-are-no-refere
{#

\let\myBib\thebibliography
\let\endmyBib\endthebibliography

\renewcommand\thebibliography[1]{\ifx\relax#1\relax\else\myBib{#1}\fi}

#}


% Stdfig -> Used as \stdfig{width}{label_name}{caption}
% Requires: image called 'caption' in img folder.
% Output: A figure with the given width, labeled as 'fig:label_name'

\newcommand{\stdfig}[3]{
    \begin{figure}
    \centering
    \includegraphics[width = #1]{#2}
    \caption{#3}
    \label{fig:#2}
    \end{figure}
}



% inplacefig -> Used as \inplacefig{width}{img_name}
% Requires: image called 'img_name' in img folder.
% Output: an inplace figure with the given width, labeled as 'fig:label_name'.

\newcommand{\inplacefig}[2]{
    \begin{figure}[H]
    \centering
    \includegraphics[width = #1]{#2}
    \label{fig:#2}
    \end{figure}
}


%\makeglossaries
%\input{special/glossary}

\begin{document}


\selectlanguage{ngerman}

\selectlanguage{english}



%% Front page.
\title{ {{ title }} }


\renewcommand{\today}{ {{ date }} }

\date{}



    \author{
        
            {{ author }} \\
        
    }


%\maketitle
\begin{titlepage}
	\begin{longtable}{p{.4\textwidth} p{.4\textwidth}}
	  {\includegraphics[height=2.6cm]{images/1und1-logo}} & 
	  {\includegraphics[height=2.6cm]{images/dhbw.png}}
	\end{longtable}
	\enlargethispage{20mm}
	\begin{center}
	  \vspace*{12mm}	{\LARGE\bf Systemnahe Programmierung }\\
	  \vspace*{12mm}	des Studienganges Angewandte Informatik\\
	  \vspace*{3mm} 	an der Dualen Hochschule Baden-Württemberg Karlsruhe\\
	  \vspace*{12mm}	{\large\bf bei Prof. Dr. Ralph Lausen}\\
	  \vspace*{12mm}	von\\
	  \vspace*{3mm} 	{\large\bf Julien Hadley-Jack\\Sebastian Dernbach}\\
	  \vspace*{12mm}	Januar 2015\\
	\end{center}
	\vfill
\end{titlepage}

\newpage

{#
\renewcommand{\thepage}{\roman{page}}
\setcounter{page}{1}

% Sperrvermerk
%!TEX root = ../praxisbericht-docker.tex

\thispagestyle{empty}
% Sperrvermerk direkt hinter Titelseite
\chapter*{ {{ snippets['sperrvermerk'] }} }
%http://www.ib.dhbw-mannheim.de/fileadmin/ms/bwl-ib/Downloads_alt/Leitfaden_31.05.pdf

\vspace*{2em}


  Der vorliegende {{ arbeit }} mit dem Titel \textit{ {{ title}} } ist mit einem Sperrvermerk versehen und wird ausschließlich zu Prüfungszwecken am Studiengang {{ studiengang }} der Dualen Hochschule Baden-Württemberg {{ dhbw_standort }} vorgelegt.
Jede Einsichtnahme und Veröffentlichung – auch von Teilen der Arbeit – bedarf der vorherigen Zustimmung durch die {{ firma }}.

  The {{ arbeit }} on hand {\itshape{} {{ title}} \/} {{ snippets['anderdh'] }} {{ dhbw_standort }} contains internal resp.\ confidential data of {{ firma }}. It is strictly forbidden, to distribute the content of this paper (including data, figures, tables, charts etc.) as a whole or in extracts, to make copies or transcripts of this paper or of parts of it, to display this paper or make it available in digital, electronic or virtual form.
Exceptional cases may be considered through permission granted in written form by the author and {{ firma }}.



% Erklärung
%!TEX root = ../praxisbericht-docker.tex

\thispagestyle{empty}

% \section*{Erklärung}
% http://www.se.dhbw-mannheim.de/fileadmin/ms/wi/dl_swm/dhbw-ma-wi-organisation-bewertung-bachelorarbeit-v2-00.pdf
\vspace*{2em}



\begin{framed}
\begin{center}
\Large\bfseries Erklärung
\end{center}

\noindent
gemäß § 5 (3) der „Studien- und Prüfungsordnung DHBW Technik“ vom 22. September 2011. \\
 

\noindent
Ich habe die vorliegende Arbeit selbstständig verfasst und keine anderen als die angegebenen 
Quellen und Hilfsmittel verwendet. 

\vspace{3cm}
\noindent
\underline{\hspace{7cm}}\hfill\underline{\hspace{6cm}}\\
Ort~~~~~~~~~~~~~~~~~~~~~~~~~Datum\hfill Unterschrift\hspace{3cm}
\end{framed}


Unless otherwise indicated in the text or references, or acknowledged
above, this thesis is entirely the product of my own scholarly work. This
thesis has not been submitted either in whole or part, for a degree at this
or any other university or institution. This is to certify that the printed
version is equivalent to the submitted electronic one.

\vspace{3em}

{{ abgabe_ort }}, {{ date }}
\vspace{4em}

{{ author }}



% Abstract
\include{special/abstract}
#}

% Inhaltsverzeichnis
\pagestyle{plain}
%für die Anzeige von Unterkapiteln im Inhaltsverzeichnis: 2
\setcounter{tocdepth}{ {{ toc_depth }} }
\tableofcontents
\newpage

{#
\renewcommand{\thepage}{\arabic{page}}
\setcounter{page}{1}
#}
\begin{acronym}[YTMMM]
\setlength{\itemsep}{-\parsep}

\acro{EPROM}{Erasable Programmable Read-Only Memory}
\acro{I/O}{Input/Output}
\acro{IDE}{Integrated Development Environment (Deutsch: Integrierte Entwicklungsumgebung)}
\acro{LED}{Light-Emitting Diode}
\acro{RAM}{Random-Access Memory}
\acro{ROM}{Read-Only Memory}
\acro{XML}{Extensible Markup Language}

\end{acronym}
%% Body start.

\include{chapters-latex/{{file}}}


% Anhang
\clearpage
\pagenumbering{Roman}

% Abbildungsverzeichnis
\cleardoublepage
\listoffigures

%Tabellenverzeichnis
%\cleardoublepage
%\listoftables

% Quellcodeverzeichnis
% \cleardoublepage
% \lstlistoflistings

% Literaturverzeichnis
\cleardoublepage
\printbibliography

% Abkürzungsverzeichnis
\cleardoublepage
%!TEX root = ../praxisbericht-docker.tex

\chapter*{Abkürzungsverzeichnis}
\phantomsection \label{listofacs}
\addcontentsline{toc}{chapter}{Abkürzungsverzeichnis}

%nur verwendete Akronyme werden letztlich im Dokument angezeigt
\begin{acronym}[YTMMM]
\setlength{\itemsep}{-\parsep}

\acro{AGPL}{Affero GNU General Public License}
\acro{API}{Application Programming Interface}
\acro{WYSIWYG}{What You See Is What You Get}
\acro{HTML}{Hypertext Meta Language}
\end{acronym}


% Glossar
%\printglossary[style=altlist,title={{ snippets['glossar'] }}]

\end{document}
