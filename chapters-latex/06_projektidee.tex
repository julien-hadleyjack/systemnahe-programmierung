%!TEX root = ../systemnahe-programmierung.tex

\chapter{Projektidee}\label{projektidee}

\begin{figure}[htbp]
\centering
\includegraphics{images/keypad.jpg}
\caption{Nummerfeld einer Alarmsicherung\footnotemark}
\end{figure}
\footnotetext{Quelle: https://www.flickr.com/photos/themillers91705/4916981638}

Bei einer Alarmsicherung meldet man sich normalerweise über ein
Nummerfeld an, um den Alarmmodus an- und auzuschalten. In diesem Projekt
wollen wir eine einfaches Nummernfeld simulieren: Wenn man eine
vordefinierte 4-stellige Ziffernfolge eingiebt, kann man den Modus der
Alarmsicherung ändern (aus-/eingeschaltet).
