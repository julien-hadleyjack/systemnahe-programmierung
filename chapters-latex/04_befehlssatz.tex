%!TEX root = ../systemnahe-programmierung.tex

\chapter{Befehlssatz}\label{befehlssatz}

Dies ist ein Ausszug aus dem Befehlssatz\footnote{(o.V.) (o.J.) alle
  Befehle der 8051-Mikrocontroller-Familie,
  http://www.self8051.de/alle-Befehle-des-8051-Mikrocontroller,13290.html,
  Einsichtnahme: 27.01.2015.}.

\begin{longtable}[c]{@{}ll@{}}
\toprule
\begin{minipage}[b]{0.25\columnwidth}\raggedright\strut
\textbf{Befehl}
\strut\end{minipage} &
\begin{minipage}[b]{0.69\columnwidth}\raggedright\strut
\textbf{Beschreibung}
\strut\end{minipage}\tabularnewline
\midrule
\endhead
\begin{minipage}[t]{0.25\columnwidth}\raggedright\strut
\texttt{ACALL\ \textless{}addr11\textgreater{}}
\strut\end{minipage} &
\begin{minipage}[t]{0.69\columnwidth}\raggedright\strut
Ruft die Subroutine an der Adresse
\texttt{\textless{}addr11\textgreater{}} auf
\strut\end{minipage}\tabularnewline
\begin{minipage}[t]{0.25\columnwidth}\raggedright\strut
\texttt{ADD\ \textless{}A\textgreater{},\textless{}Operand\textgreater{}}
\strut\end{minipage} &
\begin{minipage}[t]{0.69\columnwidth}\raggedright\strut
Addiert den \texttt{\textless{}Operand\textgreater{}} zum Inhalt des
Akkumulators \texttt{\textless{}A\textgreater{}} hinzu
\strut\end{minipage}\tabularnewline
\begin{minipage}[t]{0.25\columnwidth}\raggedright\strut
\texttt{ADDC\ \textless{}A\textgreater{},\textless{}Operand\textgreater{}}
\strut\end{minipage} &
\begin{minipage}[t]{0.69\columnwidth}\raggedright\strut
Addiert den \texttt{\textless{}Operand\textgreater{}} und das
Übertragsbit zum Inhalt des Akkumulators
\texttt{\textless{}A\textgreater{}} hinzu
\strut\end{minipage}\tabularnewline
\begin{minipage}[t]{0.25\columnwidth}\raggedright\strut
\texttt{AJMP\ \textless{}addr11\textgreater{}}
\strut\end{minipage} &
\begin{minipage}[t]{0.69\columnwidth}\raggedright\strut
Springt zu Adresse \texttt{\textless{}addr11\textgreater{}}
\strut\end{minipage}\tabularnewline
\begin{minipage}[t]{0.25\columnwidth}\raggedright\strut
\texttt{ANL\ \textless{}Zielbyte\textgreater{},\ \textless{}Quellenbyte\textgreater{}}
\strut\end{minipage} &
\begin{minipage}[t]{0.69\columnwidth}\raggedright\strut
Speichert eine bitweise logische UND-Verknüpfung zwischen dem
\texttt{\textless{}Zielbyte\textgreater{}} und dem
\texttt{\textless{}Quellenbyte\textgreater{}} im
\texttt{\textless{}Zielbyte\textgreater{}}
\strut\end{minipage}\tabularnewline
\begin{minipage}[t]{0.25\columnwidth}\raggedright\strut
\texttt{CJNE\ \textless{}Operand1\textgreater{},}
\texttt{\textless{}Operand2\textgreater{},\textless{}rel\textgreater{}}
\strut\end{minipage} &
\begin{minipage}[t]{0.69\columnwidth}\raggedright\strut
Springe zu \texttt{\textless{}rel\textgreater{}} falls die Werte
\texttt{\textless{}Operand1\textgreater{}} und
\texttt{\textless{}Operand2\textgreater{}} ungleich sind
\strut\end{minipage}\tabularnewline
\begin{minipage}[t]{0.25\columnwidth}\raggedright\strut
\texttt{CLR\ \textless{}bit\textgreater{}/\textless{}A\textgreater{}}
\strut\end{minipage} &
\begin{minipage}[t]{0.69\columnwidth}\raggedright\strut
Löscht das Bit \texttt{\textless{}bit\textgreater{}} bzw. den
Akkumulator \texttt{\textless{}A\textgreater{}}
\strut\end{minipage}\tabularnewline
\begin{minipage}[t]{0.25\columnwidth}\raggedright\strut
\texttt{CPL\ \textless{}bit\textgreater{}/\textless{}A\textgreater{}}
\strut\end{minipage} &
\begin{minipage}[t]{0.69\columnwidth}\raggedright\strut
Komplementiert das Bit \texttt{\textless{}bit\textgreater{}} bzw. den
Akkumulator \texttt{\textless{}A\textgreater{}}
\strut\end{minipage}\tabularnewline
\begin{minipage}[t]{0.25\columnwidth}\raggedright\strut
\texttt{DA\ \textless{}A\textgreater{}}
\strut\end{minipage} &
\begin{minipage}[t]{0.69\columnwidth}\raggedright\strut
Korrigiere den Dezimalwert des Akkumulators
\texttt{\textless{}A\textgreater{}} nach einer Addition
\strut\end{minipage}\tabularnewline
\begin{minipage}[t]{0.25\columnwidth}\raggedright\strut
\texttt{DEC\ \textless{}byte\textgreater{}}
\strut\end{minipage} &
\begin{minipage}[t]{0.69\columnwidth}\raggedright\strut
Dekrementiere \texttt{\textless{}byte\textgreater{}} um 1
\strut\end{minipage}\tabularnewline
\begin{minipage}[t]{0.25\columnwidth}\raggedright\strut
\texttt{DIV\ \textless{}A\textgreater{},\textless{}B\textgreater{}}
\strut\end{minipage} &
\begin{minipage}[t]{0.69\columnwidth}\raggedright\strut
Dividiere Akkumulator \texttt{\textless{}A\textgreater{}} durch Register
\texttt{\textless{}B\textgreater{}}
\strut\end{minipage}\tabularnewline
\begin{minipage}[t]{0.25\columnwidth}\raggedright\strut
\texttt{DJNZ\ \textless{}byte\textgreater{},\textless{}rel\textgreater{}}
\strut\end{minipage} &
\begin{minipage}[t]{0.69\columnwidth}\raggedright\strut
Dekrementiere \texttt{\textless{}byte\textgreater{}} um 1 und springe zu
\texttt{\textless{}rel\textgreater{}} wenn das
\texttt{\textless{}byte\textgreater{}} nicht Null ist
\strut\end{minipage}\tabularnewline
\begin{minipage}[t]{0.25\columnwidth}\raggedright\strut
\texttt{INC\ \textless{}byte\textgreater{}/\textless{}DPTR\textgreater{}}
\strut\end{minipage} &
\begin{minipage}[t]{0.69\columnwidth}\raggedright\strut
Inkrementiere \texttt{\textless{}byte\textgreater{}} bzw.
\texttt{\textless{}DPTR\textgreater{}} um 1
\strut\end{minipage}\tabularnewline
\begin{minipage}[t]{0.25\columnwidth}\raggedright\strut
\texttt{JB\ \textless{}bit\textgreater{},\textless{}rel\textgreater{}}
\strut\end{minipage} &
\begin{minipage}[t]{0.69\columnwidth}\raggedright\strut
Springe zu \texttt{\textless{}rel\textgreater{}}, wenn
\texttt{\textless{}byte\textgreater{}} gesetzt ist (=1)
\strut\end{minipage}\tabularnewline
\begin{minipage}[t]{0.25\columnwidth}\raggedright\strut
\texttt{JBC\ \textless{}bit\textgreater{},\textless{}rel\textgreater{}}
\strut\end{minipage} &
\begin{minipage}[t]{0.69\columnwidth}\raggedright\strut
Springe zu \texttt{\textless{}rel\textgreater{}}, wenn
\texttt{\textless{}byte\textgreater{}} gesetzt ist (=1) und lösche
dieses anschließend
\strut\end{minipage}\tabularnewline
\begin{minipage}[t]{0.25\columnwidth}\raggedright\strut
\texttt{JC\ \textless{}rel\textgreater{}}
\strut\end{minipage} &
\begin{minipage}[t]{0.69\columnwidth}\raggedright\strut
Springe zu \texttt{\textless{}rel\textgreater{}}, wenn das Übertragsbit
gesetzt ist (=1)
\strut\end{minipage}\tabularnewline
\begin{minipage}[t]{0.25\columnwidth}\raggedright\strut
\texttt{JMP\ \textless{}A\textgreater{}+\textless{}DPTR\textgreater{}}
\strut\end{minipage} &
\begin{minipage}[t]{0.69\columnwidth}\raggedright\strut
Addiere den Akkumulator \texttt{\textless{}A\textgreater{}} zum
Datenanzeiger \texttt{\textless{}DPTR\textgreater{}} und lade das
Ergebnis in den Programmzähler
\strut\end{minipage}\tabularnewline
\begin{minipage}[t]{0.25\columnwidth}\raggedright\strut
\texttt{JNB\ \textless{}bit\textgreater{},\textless{}rel\textgreater{}}
\strut\end{minipage} &
\begin{minipage}[t]{0.69\columnwidth}\raggedright\strut
Springe zu \texttt{\textless{}rel\textgreater{}}, wenn
\texttt{\textless{}byte\textgreater{}} nicht gesetzt ist (=0)
\strut\end{minipage}\tabularnewline
\begin{minipage}[t]{0.25\columnwidth}\raggedright\strut
\texttt{JNC\ \textless{}rel\textgreater{}}
\strut\end{minipage} &
\begin{minipage}[t]{0.69\columnwidth}\raggedright\strut
Springe zu \texttt{\textless{}rel\textgreater{}}, wenn das Übertragsbit
nicht gesetzt ist (=0)
\strut\end{minipage}\tabularnewline
\begin{minipage}[t]{0.25\columnwidth}\raggedright\strut
\texttt{JNZ\ \textless{}rel\textgreater{}}
\strut\end{minipage} &
\begin{minipage}[t]{0.69\columnwidth}\raggedright\strut
Springe zu \texttt{\textless{}rel\textgreater{}}, wenn der Akkumulator
nicht Null ist
\strut\end{minipage}\tabularnewline
\begin{minipage}[t]{0.25\columnwidth}\raggedright\strut
\texttt{JZ\ \textless{}rel\textgreater{}}
\strut\end{minipage} &
\begin{minipage}[t]{0.69\columnwidth}\raggedright\strut
Springe zu \texttt{\textless{}rel\textgreater{}}, wenn der Akkumulator
Null ist
\strut\end{minipage}\tabularnewline
\begin{minipage}[t]{0.25\columnwidth}\raggedright\strut
\texttt{LCALL\ \textless{}addr16\textgreater{}}
\strut\end{minipage} &
\begin{minipage}[t]{0.69\columnwidth}\raggedright\strut
Ruft bedingungslos Subroutine an der Adresse
\texttt{\textless{}addr16\textgreater{}} auf
\strut\end{minipage}\tabularnewline
\begin{minipage}[t]{0.25\columnwidth}\raggedright\strut
\texttt{LJMP\ \textless{}addr16\textgreater{}}
\strut\end{minipage} &
\begin{minipage}[t]{0.69\columnwidth}\raggedright\strut
Springt bedingungslos zur Adresse
\texttt{\textless{}addr16\textgreater{}}
\strut\end{minipage}\tabularnewline
\begin{minipage}[t]{0.25\columnwidth}\raggedright\strut
\texttt{MOV\ \textless{}Zielbyte\textgreater{},}
\texttt{\textless{}Quellenbyte\textgreater{}}
\strut\end{minipage} &
\begin{minipage}[t]{0.69\columnwidth}\raggedright\strut
Kopiere das \texttt{\textless{}Quellenbyte\textgreater{}} in das
\texttt{\textless{}Zielbyte\textgreater{}}
\strut\end{minipage}\tabularnewline
\begin{minipage}[t]{0.25\columnwidth}\raggedright\strut
\texttt{MOV\ \textless{}Zielbit\textgreater{},}
\texttt{\textless{}Quellenbit\textgreater{}}
\strut\end{minipage} &
\begin{minipage}[t]{0.69\columnwidth}\raggedright\strut
Kopiere das \texttt{\textless{}Quellenbit\textgreater{}} in das
\texttt{\textless{}Zielbit\textgreater{}}
\strut\end{minipage}\tabularnewline
\begin{minipage}[t]{0.25\columnwidth}\raggedright\strut
\texttt{MOV\ \textless{}DPTR\textgreater{},\textless{}data16\textgreater{}}
\strut\end{minipage} &
\begin{minipage}[t]{0.69\columnwidth}\raggedright\strut
Lade die Konstante \texttt{\textless{}data16\textgreater{}} in den
Datenzeiger \texttt{\textless{}DPTR\textgreater{}}
\strut\end{minipage}\tabularnewline
\begin{minipage}[t]{0.25\columnwidth}\raggedright\strut
\texttt{MUL\ \textless{}A\textgreater{},\textless{}B\textgreater{}}
\strut\end{minipage} &
\begin{minipage}[t]{0.69\columnwidth}\raggedright\strut
Multipliziere den Akkumulator \texttt{\textless{}A\textgreater{}} zum
Register \texttt{\textless{}B\textgreater{}}
\strut\end{minipage}\tabularnewline
\begin{minipage}[t]{0.25\columnwidth}\raggedright\strut
\texttt{NOP}
\strut\end{minipage} &
\begin{minipage}[t]{0.69\columnwidth}\raggedright\strut
Setze Programm mit folgendem Befehl fort
\strut\end{minipage}\tabularnewline
\begin{minipage}[t]{0.25\columnwidth}\raggedright\strut
\texttt{ORL\ \textless{}Zielbyte\textgreater{},}
\texttt{\textless{}Quellenbyte\textgreater{}}
\strut\end{minipage} &
\begin{minipage}[t]{0.69\columnwidth}\raggedright\strut
Speichert eine bitweise logische ODER-Verknüpfung zwischen dem
\texttt{\textless{}Zielbyte\textgreater{}} und dem
\texttt{\textless{}Quellenbyte\textgreater{}} im
\texttt{\textless{}Zielbyte\textgreater{}}
\strut\end{minipage}\tabularnewline
\begin{minipage}[t]{0.25\columnwidth}\raggedright\strut
\texttt{POP\ \textless{}byte\textgreater{}}
\strut\end{minipage} &
\begin{minipage}[t]{0.69\columnwidth}\raggedright\strut
Bringe den Wert vom Stack Pointer zum Byte
\texttt{\textless{}byte\textgreater{}} und dekrementiere den Stack
Pointer
\strut\end{minipage}\tabularnewline
\begin{minipage}[t]{0.25\columnwidth}\raggedright\strut
\texttt{PUSH\ \textless{}byte\textgreater{}}
\strut\end{minipage} &
\begin{minipage}[t]{0.69\columnwidth}\raggedright\strut
Kopiere den Wert vom Byte \texttt{\textless{}byte\textgreater{}} in den
Stack Pointer und inkrementiere den Stack Pointer
\strut\end{minipage}\tabularnewline
\begin{minipage}[t]{0.25\columnwidth}\raggedright\strut
\texttt{RET}
\strut\end{minipage} &
\begin{minipage}[t]{0.69\columnwidth}\raggedright\strut
Springt aus der Subroutine zurück
\strut\end{minipage}\tabularnewline
\begin{minipage}[t]{0.25\columnwidth}\raggedright\strut
\texttt{RETI}
\strut\end{minipage} &
\begin{minipage}[t]{0.69\columnwidth}\raggedright\strut
Springe aus dem Interrupt zurück
\strut\end{minipage}\tabularnewline
\begin{minipage}[t]{0.25\columnwidth}\raggedright\strut
\texttt{RL\ \textless{}A\textgreater{}}
\strut\end{minipage} &
\begin{minipage}[t]{0.69\columnwidth}\raggedright\strut
Schiebe die Bits des Akkumulators \texttt{\textless{}A\textgreater{}}
nach links
\strut\end{minipage}\tabularnewline
\begin{minipage}[t]{0.25\columnwidth}\raggedright\strut
\texttt{RLC\ \textless{}A\textgreater{}}
\strut\end{minipage} &
\begin{minipage}[t]{0.69\columnwidth}\raggedright\strut
Schiebe die Bits samt Übertragsbit des Akkumulators
\texttt{\textless{}A\textgreater{}} nach links
\strut\end{minipage}\tabularnewline
\begin{minipage}[t]{0.25\columnwidth}\raggedright\strut
\texttt{RR\ \textless{}A\textgreater{}}
\strut\end{minipage} &
\begin{minipage}[t]{0.69\columnwidth}\raggedright\strut
Schiebe die Bits des Akkumulators \texttt{\textless{}A\textgreater{}}
nach rechts
\strut\end{minipage}\tabularnewline
\begin{minipage}[t]{0.25\columnwidth}\raggedright\strut
\texttt{RRC\ \textless{}A\textgreater{}}
\strut\end{minipage} &
\begin{minipage}[t]{0.69\columnwidth}\raggedright\strut
Schiebe die Bits samt Übertragsbit des Akkumulators
\texttt{\textless{}A\textgreater{}} nach rechts
\strut\end{minipage}\tabularnewline
\begin{minipage}[t]{0.25\columnwidth}\raggedright\strut
\texttt{SETB\ \textless{}bit\textgreater{}}
\strut\end{minipage} &
\begin{minipage}[t]{0.69\columnwidth}\raggedright\strut
Setze das Bit \texttt{\textless{}bit\textgreater{}}
\strut\end{minipage}\tabularnewline
\begin{minipage}[t]{0.25\columnwidth}\raggedright\strut
\texttt{SJMP\ \textless{}rel\textgreater{}}
\strut\end{minipage} &
\begin{minipage}[t]{0.69\columnwidth}\raggedright\strut
Springe unbedingt zu \texttt{\textless{}rel\textgreater{}}
\strut\end{minipage}\tabularnewline
\begin{minipage}[t]{0.25\columnwidth}\raggedright\strut
\texttt{SUBB\ \textless{}A\textgreater{},\textless{}Operand\textgreater{}}
\strut\end{minipage} &
\begin{minipage}[t]{0.69\columnwidth}\raggedright\strut
Subtrahiere den Operanten \texttt{\textless{}Operand\textgreater{}} vom
Akkumulator \texttt{\textless{}A\textgreater{}}
\strut\end{minipage}\tabularnewline
\begin{minipage}[t]{0.25\columnwidth}\raggedright\strut
\texttt{SWAP\ \textless{}A\textgreater{}}
\strut\end{minipage} &
\begin{minipage}[t]{0.69\columnwidth}\raggedright\strut
Vertausche Halbbytes im Akkumulator \texttt{\textless{}A\textgreater{}}
\strut\end{minipage}\tabularnewline
\begin{minipage}[t]{0.25\columnwidth}\raggedright\strut
\texttt{XCH\ \textless{}A\textgreater{},\textless{}Byte\textgreater{}}
\strut\end{minipage} &
\begin{minipage}[t]{0.69\columnwidth}\raggedright\strut
Vertausche das Byte \texttt{\textless{}Byte\textgreater{}} im
Akkumulator \texttt{\textless{}A\textgreater{}}
\strut\end{minipage}\tabularnewline
\begin{minipage}[t]{0.25\columnwidth}\raggedright\strut
\texttt{XRL\ \textless{}Zielbyte\textgreater{},}
\texttt{\textless{}Quellenbyte\textgreater{}}
\strut\end{minipage} &
\begin{minipage}[t]{0.69\columnwidth}\raggedright\strut
Speichert eine bitweise logische Exklusiv-ODER-Verknüpfung zwischen dem
\texttt{\textless{}Zielbyte\textgreater{}} und dem
\texttt{\textless{}Quellenbyte\textgreater{}} im
\texttt{\textless{}Zielbyte\textgreater{}}
\strut\end{minipage}\tabularnewline
\bottomrule
\end{longtable}