%!TEX root = ../systemnahe-programmierung.tex

\chapter{Geschichte}\label{geschichte}

\begin{figure}[htbp]
\centering
\includegraphics[width=0.8\textwidth]{images/8051-board}
\caption{8051 Board\footnotemark}
\end{figure}
\footnotetext{8051 Primer Board, https://www.flickr.com/photos/pantechsolutions/5760938387, Einsichtsnahme: 31.01.2015}

1980 präsentierte Intel den Nachfolger des 8048, den 8051. Er war als
Erweiterung des 8048 zu sehen, wurde von Intel intern als ``Verbesserte
MCS-48 Architektur'' bezeichnent und enthielt somit jegliche Funktionen
dessen. Unteranderem wurden die Anzahl der Register mit 4 verdoppelt,
ein zweiter Timer eingeführt und diese auf 16-Bit aufgestockt.

Zum großen Erfolg des 8051 trug Intel mit den von Start ab vohandenen
nötigen Programmen (Assembler, Emulator, Software Beispielen) und der
ausführlichen Dokumentation bei. So kamen bald verschiedene Varianten
ohne \ac{ROM} (8031) oder mit \ac{EPROM} (8751) auf und wurden bald
durch noch bessere Versionen mit mehr \ac{ROM}, \ac{RAM} und Timern
ersetzt (z.B. 8052, 8kB \ac{ROM}, 256B \ac{RAM}, 3 16-Bit Timer). Durch
die Lizensierung verschiedenster Firmen zur Herstellung des 8051
entstanden immer mehr Varianten des ursprünglichen Microcontrollers. So
kamen z.B. in den 90ern Varianten mit Flash Speicher auf, welche für
Fehlerbehebungen oder neue Funktionen neu programmiert werden konnten
und somit die Kosten senkten

Mit der großen Aktzeptanz des 8051 wurden die Applikationen immer größer
und benötigten somit auch mehr Leistung. Diese sollte eine 16-Bit
Version des Controllers mit sich bringen, die kompatibel zur
ursprünglichen Version war. Nach dem Misserfolg dieser Variante konnten
erst später Dritthersteller eine erfolgreiche Variante mit mehr Megaherz
auf den Markt bringen.
