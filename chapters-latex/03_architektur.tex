%!TEX root = ../systemnahe-programmierung.tex

\chapter{Mikrocontroller-Architektur}\label{mikrocontroller-architektur}

Bei der Intel 8051 Familie handelt es sich um einen 8-Bit Prozessorkern
mit einem einheitlichen Befehlssatz. Es werden 128 Byte \ac{RAM} und
4096 Byte \ac{ROM} intern verbaut, wobei die Möglichkeit zum Anschluss
von externem \ac{RAM} und \ac{ROM} besteht. Außerdem besitzt er 2 Timer
und 4 8-bit \ac{I/O} Ports. Sie besitzt 2 externe Interrupt Quellen
sowie 2 verschiedene Interrupt Prioritäten.

Als Datenspeicher dienen die 8 Register, aufgeteilt auf die 4
Registerbänke. Diese sind direkt über ihre Adresse oder als reguläres
Register ansprechbar. Als Programmspeicher kann entweder der interne
oder der externe Speicher verwendet werden.

Durch die Trennung von Befehls- und Datenspeicher ist die Harvard
Architektur zu erkennen.
