% Warns the user when LateX is wrongly used
% http://www.ctan.org/tex-archive/macros/latex/contrib/nag
\RequirePackage[l2tabu, orthodox]{nag}

% http://www.ctan.org/pkg/scrreprt
\documentclass[
    pagesize=pdftex,    % Page size is set with \pdfpagewidth and \pdfpageheight
    twoside=false,      % Einseitiger Druck.
    fontsize=12pt,      % Schriftgroesse
    parskip=half,       % Halbe Zeile Abstand zwischen Absätzen.
    headsepline,        % Linie nach Kopfzeile.
    footsepline,        % Linie vor Fusszeile.
    abstract=false,     % Abstract Überschriften
    listof=totoc,       % show table of figures in toc but unnumbered
    toc=bibliography,   % show bibliography in toc but unnumbered
]{scrreprt}

\usepackage{special-generated/customsettings}

\title{Systemnahe Programmierung}
\author{}
\date{\today}

% Support for hypertext links
% http://www.ctan.org/pkg/hyperref
\usepackage[%
    pdftitle={Systemnahe Programmierung},                % text for PDF Title field 
    pdfauthor={},              % text for PDF Author field 
    pdfsubject={Systemnahe Programmierung},              % text for PDF Subject field 
    pdfcreator={pdflatex, LaTeX with KOMA-Script},   % text for PDF Creator field 
    pdfpagemode=UseOutlines,        % show bookmarks
    pdflang={de},                   % language identifier (RFC 3066)
    hidelinks,                      % hide links (no color, no border)
    bookmarksnumbered=true,         % put section numbers in bookmarks
    pdfdisplaydoctitle=true,        % display document title instead of file name in title bar
]{hyperref}

% A new bookmark (outline) organization for package hyperref
% bookmarks are generated on first compile
% http://www.ctan.org/pkg/bookmark
% \usepackage{bookmark}

%%% New commands.

% Don't output references in case they're empty
% http://tex.stackexchange.com/questions/74476/how-to-avoid-empty-thebibliography-environment-bibtex-if-there-are-no-refere



% Stdfig -> Used as \stdfig{width}{label_name}{caption}
% Requires: image called 'caption' in img folder.
% Output: A figure with the given width, labeled as 'fig:label_name'

\newcommand{\stdfig}[3]{
    \begin{figure}
    \centering
    \includegraphics[width = #1]{img/#2.eps}
    \caption{#3}
    \label{fig:#2}
    \end{figure}
}



% inplacefig -> Used as \inplacefig{width}{img_name}
% Requires: image called 'img_name' in img folder.
% Output: an inplace figure with the given width, labeled as 'fig:label_name'.

\newcommand{\inplacefig}[2]{
    \begin{figure}[H]
    \centering
    \includegraphics[width = #1]{img/#2.eps}
    \label{fig:#2}
    \end{figure}
}


%\makeglossaries
%\input{special/glossary}

\begin{document}


\selectlanguage{ngerman}



%% Front page.
\title{Systemnahe Programmierung}



\author{
        Julien Hadley Jack \\
        Sebastian Dernbach \\
       }


%\maketitle
%!TEX root = ../dokumentation.tex

\begin{titlepage}
	\begin{longtable}{p{.4\textwidth} p{.4\textwidth}}
	  {\includegraphics[height=2.6cm]{images/1und1-logo}} & 
	  {\includegraphics[height=2.6cm]{images/dhbw.png}}
	\end{longtable}
	\enlargethispage{20mm}
	\begin{center}
	  \vspace*{12mm}	{\LARGE\bf Systemnahe Programmierung }\\
	  \vspace*{12mm}	{\large\bf bei Prof. Dr. Ralph Lausen}\\
%	  \vspace*{12mm}	für die Prüfung zum\\
%	  \vspace*{3mm} 	{\bf \abschluss}\\
	  \vspace*{12mm}	des Studienganges Angewandte Informatik\\
	  \vspace*{3mm} 	an der Dualen Hochschule Baden-Württemberg Karlsruhe\\
	  \vspace*{12mm}	von\\
	  \vspace*{3mm} 	{\large\bf Julien Hadley-Jack\\Sebastian Dernbach}\\
	  \vspace*{12mm}	Januar 2015\\
	\end{center}
	\vfill
\end{titlepage}

\newpage



% Inhaltsverzeichnis
\pagestyle{plain}
%für die Anzeige von Unterkapiteln im Inhaltsverzeichnis: 2
\setcounter{tocdepth}{2}
\tableofcontents
\newpage



%% Body start.
%!TEX root = ../systemnahe-programmierung.tex

\chapter{Einleitung}\label{einleitung}

\begin{figure}[htbp]
\centering
\includegraphics{images/washing-maschine}
\caption[Microcontroller im Alltag]{Microcontroller im Alltag\footnotemark{}}
\end{figure}
\footnotetext{Choose only one, https://www.flickr.com/photos/97922031@N05/15884820461,
  Einsichtsnahme: 31.01.2015}

Systemnahe Programmierung beschäftigt sich mit der Erstellung beziehungsweise Programmierung von
Software, die Teil des Betriebssystemes oder sehr eng mit diesem verbunden ist.

Die Software dient hierbei als Abstraktionsschicht zwischen Hardware und Betriebssystem, welche
leichten Zugriff auf einfache Funktionen des Systems bietet.

Heutzutage nimmt Systemnahe Programmierung einen sehr hohen Stellenwert ein. Sie ist mittlerweile in
unserem Alltag in Form von Microcontrollern mit verschiedensten Einsatzgebieten vertreten.
Microcontroller sind in Autos, Mobiltelefonen, Waschmaschinen, elektrischen Zahnbürsten,
Fernbedienungen und vielen anderen Geräten zu finden.

Die Systeme sind nicht nur omnipräsent sondern der Markt wächst immer mehr. Durch den Aufstieg von
Wearables und dem Drang alle Geräte miteinander zu verbinden, werden immer mehr Microcontroller
verbaut. So kam auch der Anstieg von 12\% in 2014 zu Stande, der für 2015 mit 15\% noch höher
ausfallen soll\footnote{(o.V.) (2015) Microcontroller market 2015,
  http://www.emittsolutions.com/section/market-analysis/market\_analysis\_microcontroller.html,
  Einsichtnahme: 27.01.2015.}.
%!TEX root = ../systemnahe-programmierung.tex

\chapter{Geschichte}\label{geschichte}

\begin{figure}[htbp]
\centering
\includegraphics[width=0.8\textwidth]{images/8051-board}
\caption{8051 Board\footnotemark}
\end{figure}
\footnotetext{8051 Primer Board, https://www.flickr.com/photos/pantechsolutions/5760938387, Einsichtsnahme: 31.01.2015}

1980 präsentierte Intel den Nachfolger des 8048, den 8051. Er war als
Erweiterung des 8048 zu sehen, wurde von Intel intern als ``Verbesserte
MCS-48 Architektur'' bezeichnent und enthielt somit jegliche Funktionen
dessen. Unteranderem wurden die Anzahl der Register mit 4 verdoppelt,
ein zweiter Timer eingeführt und diese auf 16-Bit aufgestockt.

Zum großen Erfolg des 8051 trug Intel mit den von Start ab vohandenen
nötigen Programmen (Assembler, Emulator, Software Beispielen) und der
ausführlichen Dokumentation bei. So kamen bald verschiedene Varianten
ohne \ac{ROM} (8031) oder mit \ac{EPROM} (8751) auf und wurden bald
durch noch bessere Versionen mit mehr \ac{ROM}, \ac{RAM} und Timern
ersetzt (z.B. 8052, 8kB \ac{ROM}, 256B \ac{RAM}, 3 16-Bit Timer). Durch
die Lizensierung verschiedenster Firmen zur Herstellung des 8051
entstanden immer mehr Varianten des ursprünglichen Microcontrollers. So
kamen z.B. in den 90ern Varianten mit Flash Speicher auf, welche für
Fehlerbehebungen oder neue Funktionen neu programmiert werden konnten
und somit die Kosten senkten

Mit der großen Aktzeptanz des 8051 wurden die Applikationen immer größer
und benötigten somit auch mehr Leistung. Diese sollte eine 16-Bit
Version des Controllers mit sich bringen, die kompatibel zur
ursprünglichen Version war. Nach dem Misserfolg dieser Variante konnten
erst später Dritthersteller eine erfolgreiche Variante mit mehr Megaherz
auf den Markt bringen.

%!TEX root = ../systemnahe-programmierung.tex

\chapter{Mikrocontroller-Architektur}\label{mikrocontroller-architektur}

Bei der Intel 8051 Familie handelt es sich um einen 8-Bit Prozessorkern mit einem einheitlichen
Befehlssatz. Es werden 128 Byte \ac{RAM} und 4096 Byte \ac{ROM} intern verbaut, wobei die
Möglichkeit zum Anschluss von externem \ac{RAM} und \ac{ROM} besteht. Außerdem besitzt er 2 Timer
und 4 8-bit \ac{I/O} Ports. Sie besitzt 2 externe Interrupt Quellen sowie 2 verschiedene Interrupt
Prioritäten.

Als Datenspeicher dienen die 8 Register, aufgeteilt auf die 4 Registerbänke. Diese sind direkt über
ihre Adresse oder als reguläres Register ansprechbar. Als Programmspeicher kann entweder der interne
oder der externe Speicher verwendet werden.

Zur Ausführung eines Befehls benötigt der 8051 mindestens 12 Takte. Durch die Trennung von Befehls-
und Datenspeicher ist die Harvard Architektur zu erkennen.
%!TEX root = ../systemnahe-programmierung.tex

\chapter{Befehlssatz}\label{befehlssatz}

Dies ist ein Ausszug aus dem Befehlssatz\footnote{(o.V.) (o.J.) alle Befehle der
  8051-Mikrocontroller-Familie,
  http://www.self8051.de/alle-Befehle-des-8051-Mikrocontroller,13290.html, Einsichtnahme:
  27.01.2015.}.

\begin{longtable}[c]{@{}ll@{}}
\toprule
\begin{minipage}[b]{0.25\columnwidth}\raggedright\strut
\textbf{Befehl}
\strut\end{minipage} & \begin{minipage}[b]{0.69\columnwidth}\raggedright\strut
\textbf{Beschreibung}
\strut\end{minipage}\tabularnewline
\midrule
\endhead
\begin{minipage}[t]{0.25\columnwidth}\raggedright\strut
\mintinline{text}{ACALL <addr11>}
\strut\end{minipage} & \begin{minipage}[t]{0.69\columnwidth}\raggedright\strut
Ruft die Subroutine an der Adresse \mintinline{text}{<addr11>} auf
\strut\end{minipage}\tabularnewline
\begin{minipage}[t]{0.25\columnwidth}\raggedright\strut
\mintinline{text}{ADD <A>,<Operand>}
\strut\end{minipage} & \begin{minipage}[t]{0.69\columnwidth}\raggedright\strut
Addiert den \mintinline{text}{<Operand>} zum Inhalt des Akkumulators \mintinline{text}{<A>} hinzu
\strut\end{minipage}\tabularnewline
\begin{minipage}[t]{0.25\columnwidth}\raggedright\strut
\mintinline{text}{ADDC <A>,<Operand>}
\strut\end{minipage} & \begin{minipage}[t]{0.69\columnwidth}\raggedright\strut
Addiert den \mintinline{text}{<Operand>} und das Übertragsbit zum Inhalt des Akkumulators \mintinline{text}{<A>}
hinzu
\strut\end{minipage}\tabularnewline
\begin{minipage}[t]{0.25\columnwidth}\raggedright\strut
\mintinline{text}{AJMP <addr11>}
\strut\end{minipage} & \begin{minipage}[t]{0.69\columnwidth}\raggedright\strut
Springt zu Adresse \mintinline{text}{<addr11>}
\strut\end{minipage}\tabularnewline
\begin{minipage}[t]{0.25\columnwidth}\raggedright\strut
\mintinline{text}{ANL <Zielbyte>, <Quellenbyte>}
\strut\end{minipage} & \begin{minipage}[t]{0.69\columnwidth}\raggedright\strut
Speichert eine bitweise logische UND-Verknüpfung zwischen dem \mintinline{text}{<Zielbyte>} und dem
\mintinline{text}{<Quellenbyte>} im \mintinline{text}{<Zielbyte>}
\strut\end{minipage}\tabularnewline
\begin{minipage}[t]{0.25\columnwidth}\raggedright\strut
\mintinline{text}{CJNE <Operand1>,} \mintinline{text}{<Operand2>,<rel>}
\strut\end{minipage} & \begin{minipage}[t]{0.69\columnwidth}\raggedright\strut
Springe zu \mintinline{text}{<rel>} falls die Werte \mintinline{text}{<Operand1>} und \mintinline{text}{<Operand2>}
ungleich sind
\strut\end{minipage}\tabularnewline
\begin{minipage}[t]{0.25\columnwidth}\raggedright\strut
\mintinline{text}{CLR <bit>/<A>}
\strut\end{minipage} & \begin{minipage}[t]{0.69\columnwidth}\raggedright\strut
Löscht das Bit \mintinline{text}{<bit>} bzw. den Akkumulator \mintinline{text}{<A>}
\strut\end{minipage}\tabularnewline
\begin{minipage}[t]{0.25\columnwidth}\raggedright\strut
\mintinline{text}{CPL <bit>/<A>}
\strut\end{minipage} & \begin{minipage}[t]{0.69\columnwidth}\raggedright\strut
Komplementiert das Bit \mintinline{text}{<bit>} bzw. den Akkumulator \mintinline{text}{<A>}
\strut\end{minipage}\tabularnewline
\begin{minipage}[t]{0.25\columnwidth}\raggedright\strut
\mintinline{text}{DA <A>}
\strut\end{minipage} & \begin{minipage}[t]{0.69\columnwidth}\raggedright\strut
Korrigiere den Dezimalwert des Akkumulators \mintinline{text}{<A>} nach einer Addition
\strut\end{minipage}\tabularnewline
\begin{minipage}[t]{0.25\columnwidth}\raggedright\strut
\mintinline{text}{DEC <byte>}
\strut\end{minipage} & \begin{minipage}[t]{0.69\columnwidth}\raggedright\strut
Dekrementiere \mintinline{text}{<byte>} um 1
\strut\end{minipage}\tabularnewline
\begin{minipage}[t]{0.25\columnwidth}\raggedright\strut
\mintinline{text}{DIV <A>,<B>}
\strut\end{minipage} & \begin{minipage}[t]{0.69\columnwidth}\raggedright\strut
Dividiere Akkumulator \mintinline{text}{<A>} durch Register \mintinline{text}{<B>}
\strut\end{minipage}\tabularnewline
\begin{minipage}[t]{0.25\columnwidth}\raggedright\strut
\mintinline{text}{DJNZ <byte>,<rel>}
\strut\end{minipage} & \begin{minipage}[t]{0.69\columnwidth}\raggedright\strut
Dekrementiere \mintinline{text}{<byte>} um 1 und springe zu \mintinline{text}{<rel>} wenn das \mintinline{text}{<byte>}
nicht Null ist
\strut\end{minipage}\tabularnewline
\begin{minipage}[t]{0.25\columnwidth}\raggedright\strut
\mintinline{text}{INC <byte>/<DPTR>}
\strut\end{minipage} & \begin{minipage}[t]{0.69\columnwidth}\raggedright\strut
Inkrementiere \mintinline{text}{<byte>} bzw. \mintinline{text}{<DPTR>} um 1
\strut\end{minipage}\tabularnewline
\begin{minipage}[t]{0.25\columnwidth}\raggedright\strut
\mintinline{text}{JB <bit>,<rel>}
\strut\end{minipage} & \begin{minipage}[t]{0.69\columnwidth}\raggedright\strut
Springe zu \mintinline{text}{<rel>}, wenn \mintinline{text}{<byte>} gesetzt ist (=1)
\strut\end{minipage}\tabularnewline
\begin{minipage}[t]{0.25\columnwidth}\raggedright\strut
\mintinline{text}{JBC <bit>,<rel>}
\strut\end{minipage} & \begin{minipage}[t]{0.69\columnwidth}\raggedright\strut
Springe zu \mintinline{text}{<rel>}, wenn \mintinline{text}{<byte>} gesetzt ist (=1) und lösche dieses
anschließend
\strut\end{minipage}\tabularnewline
\begin{minipage}[t]{0.25\columnwidth}\raggedright\strut
\mintinline{text}{JC <rel>}
\strut\end{minipage} & \begin{minipage}[t]{0.69\columnwidth}\raggedright\strut
Springe zu \mintinline{text}{<rel>}, wenn das Übertragsbit gesetzt ist (=1)
\strut\end{minipage}\tabularnewline
\begin{minipage}[t]{0.25\columnwidth}\raggedright\strut
\mintinline{text}{JMP <A>+<DPTR>}
\strut\end{minipage} & \begin{minipage}[t]{0.69\columnwidth}\raggedright\strut
Addiere den Akkumulator \mintinline{text}{<A>} zum Datenanzeiger \mintinline{text}{<DPTR>} und lade das Ergebnis
in den Programmzähler
\strut\end{minipage}\tabularnewline
\begin{minipage}[t]{0.25\columnwidth}\raggedright\strut
\mintinline{text}{JNB <bit>,<rel>}
\strut\end{minipage} & \begin{minipage}[t]{0.69\columnwidth}\raggedright\strut
Springe zu \mintinline{text}{<rel>}, wenn \mintinline{text}{<byte>} nicht gesetzt ist (=0)
\strut\end{minipage}\tabularnewline
\begin{minipage}[t]{0.25\columnwidth}\raggedright\strut
\mintinline{text}{JNC <rel>}
\strut\end{minipage} & \begin{minipage}[t]{0.69\columnwidth}\raggedright\strut
Springe zu \mintinline{text}{<rel>}, wenn das Übertragsbit nicht gesetzt ist (=0)
\strut\end{minipage}\tabularnewline
\begin{minipage}[t]{0.25\columnwidth}\raggedright\strut
\mintinline{text}{JNZ <rel>}
\strut\end{minipage} & \begin{minipage}[t]{0.69\columnwidth}\raggedright\strut
Springe zu \mintinline{text}{<rel>}, wenn der Akkumulator nicht Null ist
\strut\end{minipage}\tabularnewline
\begin{minipage}[t]{0.25\columnwidth}\raggedright\strut
\mintinline{text}{JZ <rel>}
\strut\end{minipage} & \begin{minipage}[t]{0.69\columnwidth}\raggedright\strut
Springe zu \mintinline{text}{<rel>}, wenn der Akkumulator Null ist
\strut\end{minipage}\tabularnewline
\begin{minipage}[t]{0.25\columnwidth}\raggedright\strut
\mintinline{text}{LCALL <addr16>}
\strut\end{minipage} & \begin{minipage}[t]{0.69\columnwidth}\raggedright\strut
Ruft bedingungslos Subroutine an der Adresse \mintinline{text}{<addr16>} auf
\strut\end{minipage}\tabularnewline
\begin{minipage}[t]{0.25\columnwidth}\raggedright\strut
\mintinline{text}{LJMP <addr16>}
\strut\end{minipage} & \begin{minipage}[t]{0.69\columnwidth}\raggedright\strut
Springt bedingungslos zur Adresse \mintinline{text}{<addr16>}
\strut\end{minipage}\tabularnewline
\begin{minipage}[t]{0.25\columnwidth}\raggedright\strut
\mintinline{text}{MOV <Zielbyte>,} \mintinline{text}{<Quellenbyte>}
\strut\end{minipage} & \begin{minipage}[t]{0.69\columnwidth}\raggedright\strut
Kopiere das \mintinline{text}{<Quellenbyte>} in das \mintinline{text}{<Zielbyte>}
\strut\end{minipage}\tabularnewline
\begin{minipage}[t]{0.25\columnwidth}\raggedright\strut
\mintinline{text}{MOV <Zielbit>,} \mintinline{text}{<Quellenbit>}
\strut\end{minipage} & \begin{minipage}[t]{0.69\columnwidth}\raggedright\strut
Kopiere das \mintinline{text}{<Quellenbit>} in das \mintinline{text}{<Zielbit>}
\strut\end{minipage}\tabularnewline
\begin{minipage}[t]{0.25\columnwidth}\raggedright\strut
\mintinline{text}{MOV <DPTR>,<data16>}
\strut\end{minipage} & \begin{minipage}[t]{0.69\columnwidth}\raggedright\strut
Lade die Konstante \mintinline{text}{<data16>} in den Datenzeiger \mintinline{text}{<DPTR>}
\strut\end{minipage}\tabularnewline
\begin{minipage}[t]{0.25\columnwidth}\raggedright\strut
\mintinline{text}{MUL <A>,<B>}
\strut\end{minipage} & \begin{minipage}[t]{0.69\columnwidth}\raggedright\strut
Multipliziere den Akkumulator \mintinline{text}{<A>} zum Register \mintinline{text}{<B>}
\strut\end{minipage}\tabularnewline
\begin{minipage}[t]{0.25\columnwidth}\raggedright\strut
\mintinline{text}{NOP}
\strut\end{minipage} & \begin{minipage}[t]{0.69\columnwidth}\raggedright\strut
Setze Programm mit folgendem Befehl fort
\strut\end{minipage}\tabularnewline
\begin{minipage}[t]{0.25\columnwidth}\raggedright\strut
\mintinline{text}{ORL <Zielbyte>,} \mintinline{text}{<Quellenbyte>}
\strut\end{minipage} & \begin{minipage}[t]{0.69\columnwidth}\raggedright\strut
Speichert eine bitweise logische ODER-Verknüpfung zwischen dem \mintinline{text}{<Zielbyte>} und dem
\mintinline{text}{<Quellenbyte>} im \mintinline{text}{<Zielbyte>}
\strut\end{minipage}\tabularnewline
\begin{minipage}[t]{0.25\columnwidth}\raggedright\strut
\mintinline{text}{POP <byte>}
\strut\end{minipage} & \begin{minipage}[t]{0.69\columnwidth}\raggedright\strut
Bringe den Wert vom Stack Pointer zum Byte \mintinline{text}{<byte>} und dekrementiere den Stack Pointer
\strut\end{minipage}\tabularnewline
\begin{minipage}[t]{0.25\columnwidth}\raggedright\strut
\mintinline{text}{PUSH <byte>}
\strut\end{minipage} & \begin{minipage}[t]{0.69\columnwidth}\raggedright\strut
Kopiere den Wert vom Byte \mintinline{text}{<byte>} in den Stack Pointer und inkrementiere den Stack
Pointer
\strut\end{minipage}\tabularnewline
\begin{minipage}[t]{0.25\columnwidth}\raggedright\strut
\mintinline{text}{RET}
\strut\end{minipage} & \begin{minipage}[t]{0.69\columnwidth}\raggedright\strut
Springt aus der Subroutine zurück
\strut\end{minipage}\tabularnewline
\begin{minipage}[t]{0.25\columnwidth}\raggedright\strut
\mintinline{text}{RETI}
\strut\end{minipage} & \begin{minipage}[t]{0.69\columnwidth}\raggedright\strut
Springe aus dem Interrupt zurück
\strut\end{minipage}\tabularnewline
\begin{minipage}[t]{0.25\columnwidth}\raggedright\strut
\mintinline{text}{RL <A>}
\strut\end{minipage} & \begin{minipage}[t]{0.69\columnwidth}\raggedright\strut
Schiebe die Bits des Akkumulators \mintinline{text}{<A>} nach links
\strut\end{minipage}\tabularnewline
\begin{minipage}[t]{0.25\columnwidth}\raggedright\strut
\mintinline{text}{RLC <A>}
\strut\end{minipage} & \begin{minipage}[t]{0.69\columnwidth}\raggedright\strut
Schiebe die Bits samt Übertragsbit des Akkumulators \mintinline{text}{<A>} nach links
\strut\end{minipage}\tabularnewline
\begin{minipage}[t]{0.25\columnwidth}\raggedright\strut
\mintinline{text}{RR <A>}
\strut\end{minipage} & \begin{minipage}[t]{0.69\columnwidth}\raggedright\strut
Schiebe die Bits des Akkumulators \mintinline{text}{<A>} nach rechts
\strut\end{minipage}\tabularnewline
\begin{minipage}[t]{0.25\columnwidth}\raggedright\strut
\mintinline{text}{RRC <A>}
\strut\end{minipage} & \begin{minipage}[t]{0.69\columnwidth}\raggedright\strut
Schiebe die Bits samt Übertragsbit des Akkumulators \mintinline{text}{<A>} nach rechts
\strut\end{minipage}\tabularnewline
\begin{minipage}[t]{0.25\columnwidth}\raggedright\strut
\mintinline{text}{SETB <bit>}
\strut\end{minipage} & \begin{minipage}[t]{0.69\columnwidth}\raggedright\strut
Setze das Bit \mintinline{text}{<bit>}
\strut\end{minipage}\tabularnewline
\begin{minipage}[t]{0.25\columnwidth}\raggedright\strut
\mintinline{text}{SJMP <rel>}
\strut\end{minipage} & \begin{minipage}[t]{0.69\columnwidth}\raggedright\strut
Springe unbedingt zu \mintinline{text}{<rel>}
\strut\end{minipage}\tabularnewline
\begin{minipage}[t]{0.25\columnwidth}\raggedright\strut
\mintinline{text}{SUBB <A>,<Operand>}
\strut\end{minipage} & \begin{minipage}[t]{0.69\columnwidth}\raggedright\strut
Subtrahiere den Operanten \mintinline{text}{<Operand>} vom Akkumulator \mintinline{text}{<A>}
\strut\end{minipage}\tabularnewline
\begin{minipage}[t]{0.25\columnwidth}\raggedright\strut
\mintinline{text}{SWAP <A>}
\strut\end{minipage} & \begin{minipage}[t]{0.69\columnwidth}\raggedright\strut
Vertausche Halbbytes im Akkumulator \mintinline{text}{<A>}
\strut\end{minipage}\tabularnewline
\begin{minipage}[t]{0.25\columnwidth}\raggedright\strut
\mintinline{text}{XCH <A>,<Byte>}
\strut\end{minipage} & \begin{minipage}[t]{0.69\columnwidth}\raggedright\strut
Vertausche das Byte \mintinline{text}{<Byte>} im Akkumulator \mintinline{text}{<A>}
\strut\end{minipage}\tabularnewline
\begin{minipage}[t]{0.25\columnwidth}\raggedright\strut
\mintinline{text}{XRL <Zielbyte>,} \mintinline{text}{<Quellenbyte>}
\strut\end{minipage} & \begin{minipage}[t]{0.69\columnwidth}\raggedright\strut
Speichert eine bitweise logische Exklusiv-ODER-Verknüpfung zwischen dem \mintinline{text}{<Zielbyte>} und
dem \mintinline{text}{<Quellenbyte>} im \mintinline{text}{<Zielbyte>}
\strut\end{minipage}\tabularnewline
\bottomrule
\end{longtable}
\include{chapters-latex/05_entwicklungsumgebung}
%!TEX root = ../systemnahe-programmierung.tex

\chapter{Projektidee}\label{projektidee}

\begin{figure}[htbp]
\centering
\includegraphics{images/keypad.jpg}
\caption{Nummerfeld einer Alarmsicherung\footnotemark}
\end{figure}
\footnotetext{Quelle: https://www.flickr.com/photos/themillers91705/4916981638}

Bei einer Alarmsicherung meldet man sich normalerweise über ein
Nummerfeld an, um den Alarmmodus an- und auzuschalten. In diesem Projekt
wollen wir eine einfaches Nummernfeld simulieren: Wenn man eine
vordefinierte 4-stellige Ziffernfolge eingiebt, kann man den Modus der
Alarmsicherung ändern (aus-/eingeschaltet).

%!TEX root = ../systemnahe-programmierung.tex

\chapter{Implementierung}\label{implementierung}

\begin{figure}[htbp]
\centering
\includegraphics{images/keypad-screenshot}
\caption{Konfiguration des Nummernfeldes}
\end{figure}

Für die Simulation des Nummernfeldes wird das \emph{Matrix Keypad} der
\emph{MCU 8051 IDE} verwendet. Als Erstes muss die Pin-Belegung
konfiguriert werden. Im Bild kann man sehen, dass das Nummerfeld auf
Port 1 erreichbar ist. Diese Einstellung kann auch in die IDE import
werden (siehe Anhang).

\begin{Shaded}
\begin{Highlighting}[]
\NormalTok{keypad      }\DataTypeTok{equ} \NormalTok{P1              }\CommentTok{;Matrix keypad}
\NormalTok{col1        }\DataTypeTok{equ} \NormalTok{keypad}\FloatTok{.3}        \CommentTok{;Column 1}
\NormalTok{col2        }\DataTypeTok{equ} \NormalTok{keypad}\FloatTok{.4}        \CommentTok{;Column 2}
\NormalTok{col3        }\DataTypeTok{equ} \NormalTok{keypad}\FloatTok{.5}        \CommentTok{;Column 3}

\NormalTok{value       }\DataTypeTok{equ} \NormalTok{30H             }\CommentTok{;Value of pressed button}
\NormalTok{pressed     bit 00H             }\CommentTok{;Was the button just pressed?}
\NormalTok{secure_mode bit}\BaseNTok{ 01h             }\CommentTok{;Is the user logiged in?}
\end{Highlighting}
\end{Shaded}

Zusätzlich zu den Bits für das Nummernfeld werden auch noch drei
Variablen definiert:

\begin{itemize}
\itemsep1pt\parskip0pt\parsep0pt
\item
  \texttt{value}: Wert der gedrückten Taste
\item
  \texttt{pressed}: Ob schon eine Taste gedrückt wurde (wird
  zurückgesetzt, wenn auf ein neuen Tastendruck gewartet wird)
\item
  \texttt{secure\_mode}: In welchem Zustand die Alarmsicherung ist (0
  für ausgeschaltet, 1 für eingeschaltet)
\end{itemize}

\begin{Shaded}
\begin{Highlighting}[]
\FunctionTok{get_button:}
    \NormalTok{clr pressed}

    \CommentTok{;Check first row}
    \KeywordTok{mov} \NormalTok{value,#}\DecValTok{1}                \CommentTok{;Start value is first number on row}
    \KeywordTok{mov} \NormalTok{keypad, #11111110B      }\CommentTok{;Mark first row}
    \NormalTok{acall check_col1            }\CommentTok{;Check all columns }

    \CommentTok{;If button was pressed in row, jump out of function}
    \KeywordTok{jb} \NormalTok{pressed, found_button    }
 
    \KeywordTok{mov} \NormalTok{value,#}\DecValTok{4}
    \KeywordTok{mov} \NormalTok{keypad, #11111101B}
    \NormalTok{acall check_col1}
 
    \KeywordTok{jb} \NormalTok{pressed, found_button}
 
    \KeywordTok{mov} \NormalTok{value,#}\DecValTok{7}
    \KeywordTok{mov} \NormalTok{keypad, #11111011B}
    \NormalTok{acall check_col1}
 
    \KeywordTok{jb} \NormalTok{pressed, found_button}

    \KeywordTok{jmp} \NormalTok{get_button}
\end{Highlighting}
\end{Shaded}

Die Zeilen des Nummernfeldes werden nacheinander überprüft. Dabei wird
zuerst \texttt{value} auf den Wert der ersten Taste aus der Reihe
gesetzt. Danach werden alle Reihen außer die ausgewählte auf \texttt{1}
gesetzt und zur Funktion \texttt{check\_col1} gesprungen, die die
Spaltenüberprüfung für die ausgewählte Reihe startet. Nach jeder Reihe
wird überprüft, ob schon eine Taste gedrückt wird. Ist dies der Fall,
dann wird aus der Funktion gesprungen. Wenn alle Reihen überprüft wurden
und kein Tastendruck erkannt wurde, wird die Überprüfung wieder von vorne
begonnen.

\begin{Shaded}
\begin{Highlighting}[]
\FunctionTok{check_col1:}
    \CommentTok{;If button wasn't pressed, jump to next colum}
    \KeywordTok{jb} \NormalTok{col1, check_col2}

    \CommentTok{;If button was pressed, wait for end of button press}
    \KeywordTok{jnb} \NormalTok{col1,}\DecValTok{$}

    \CommentTok{;Set bit that key was pressed}
    \KeywordTok{setb} \NormalTok{pressed}
    \KeywordTok{ret}
\end{Highlighting}
\end{Shaded}

Wenn in der ersten Spalte kein Tastendruck entdeckt worden ist, wird zu
\texttt{check\_col2} gesprungen, die die nächste Spalte überprüft.
Sollte die Taste aber gedrückt sein, dass muss erst auf das Ende des
Tastendrucks gewartet werden. Dafür wird einfach zur gleichen Zeile der
Überprüfung gesprungen. Dach wird noch das Bit für \texttt{pressed} auf
\texttt{1} gesetzt.

\begin{Shaded}
\begin{Highlighting}[]
\FunctionTok{check_col2:}
    \KeywordTok{jb} \NormalTok{col2, check_col3}
    \KeywordTok{jnb} \NormalTok{col2,}\DecValTok{$}
    \KeywordTok{inc} \NormalTok{value           }\CommentTok{;Increment the start value from row}
    \KeywordTok{setb} \NormalTok{pressed}
    \KeywordTok{ret}
\end{Highlighting}
\end{Shaded}

Die Überprüfung der nächsten Spalten funktioniert fast identisch. Der
einzige Unterschied findet bei einem erfolgreichem Tastendruck statt. Da
der Wert der Taste verschiedene Werte hat, muss der Inhalt von
\texttt{value} noch angepasst werden. Dafür wird einfach um den
Unterschied inkrementiert (in diesem Fall 1).

\begin{Shaded}
\begin{Highlighting}[]
\FunctionTok{check_pin:}
    \CommentTok{;Check first pin (4)}
    \NormalTok{acall get_button}
    \KeywordTok{mov} \NormalTok{A, value}
    \KeywordTok{cjne} \NormalTok{A, #}\DecValTok{4}\NormalTok{, check_pin}

    \CommentTok{;Check second pin (2)}
    \NormalTok{acall get_button}
    \KeywordTok{mov} \NormalTok{A, value}
    \KeywordTok{cjne} \NormalTok{A, #}\DecValTok{2}\NormalTok{, check_pin}

    \CommentTok{;Check third pin (6)}
    \NormalTok{acall get_button}
    \KeywordTok{mov} \NormalTok{A, value}
    \KeywordTok{cjne} \NormalTok{A, #}\DecValTok{6}\NormalTok{, check_pin}

    \CommentTok{;Check fourth pin (8)}
    \NormalTok{acall get_button}
    \KeywordTok{mov} \NormalTok{A, value}
    \KeywordTok{cjne} \NormalTok{A, #}\DecValTok{8}\NormalTok{, check_pin}

    \CommentTok{;Toggle secure mode of the system}
    \NormalTok{cpl secure_mode}
\end{Highlighting}
\end{Shaded}

Die Hauptfunktion ist \texttt{check\_pin}. Sie ruft viermal die
\texttt{get\_button}-Funktion auf und überprüft, ob der gelieferte
Tastendruck dem gewünschtem gleicht. Sollte dies einmal nicht der Fall
sein, dann wird die Suche von vorne angefangen. Aber wurde die
Ziffernfolge erfolgreich eingegeben, dann wird der Alarm-Modus
gewechselt. Die Zifferfolge lässt sich sehr leicht ändern, unter anderem
auch in der Länge.



% Anhang
\clearpage
\pagenumbering{Roman}

% Abbildungsverzeichnis
\cleardoublepage
\listoffigures

%Tabellenverzeichnis
%\cleardoublepage
%\listoftables

% Quellcodeverzeichnis
% \cleardoublepage
% \lstlistoflistings

% Literaturverzeichnis
\cleardoublepage
\printbibliography

% Abkürzungsverzeichnis
\cleardoublepage
%!TEX root = ../praxisbericht-docker.tex

\chapter*{Abkürzungsverzeichnis}
\phantomsection \label{listofacs}
\addcontentsline{toc}{chapter}{Abkürzungsverzeichnis}

%nur verwendete Akronyme werden letztlich im Dokument angezeigt
\begin{acronym}[YTMMM]
\setlength{\itemsep}{-\parsep}

\acro{AGPL}{Affero GNU General Public License}
\acro{API}{Application Programming Interface}
\acro{WYSIWYG}{What You See Is What You Get}
\acro{HTML}{Hypertext Meta Language}
\end{acronym}


% Glossar
\printglossary[style=altlist,title=Glossar]

\end{document}